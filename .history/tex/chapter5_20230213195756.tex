% !TeX root=main.tex
\chapter{بحث و نتیجه‌گیری}
%\thispagestyle{empty} 
\section{بررسی رفتارهای مشاهده شده }
همانطور که در بخش نتایج ذکر شد،
$\ValidParticipantsWithBogusPhoneSize$
نفر  از شرکت‌کنندگان
بخش های مختلف آزمایش را به درستی انجام داده بودند اما شماره تلفن نامعتر
برای فرد معرفی شده در بخش‌
\textit{معرفی اولیه}
و
\textit{معرفی ثانویه}
وارد کرده بودند. با توجه به اینکه این $\ValidParticipantsWithBogusPhoneSize$
نفر نام
خود را در قسمت معرفی وارد کرده بودند، که متوجه منظور بخش‌های مذکور نشده اند.
$1$
نفر دیگر نیز به جای شماره تلفن، ایمیل خود را وارد کرده بود که به نظر می رسد دچار خطای مشابه
شده باشد.
% ^ %%%%%%%%%%%%%%%%%%%%%%%%%%%%%%%%%%%%%%%%%%%%%%%%%%%%%%%%%%
% ^ %%%%%%%%%%%%%%%%%%%%%%%%%%%%%%%%%%%%%%%%%%%%%%%%%%%%%%%%%%
% ^ %%%%%%%%%%%%%%%%%%%%%%%%%%%%%%%%%%%%%%%%%%%%%%%%%%%%%%%%%%
% \subsection{نتایج مربوط به سه‌گانه تاریک}
\section{نتایج مربوط به سه‌گانه تاریک}
پژوهش‌های پیشین میان ویژگی‌های تاریک و رفتار
به اشتراک گذاری اطلاعات خصوصی دیگران به شکل
\textit{
    شایعه پراکنی
}
\LTRfootnote{
    Gossip
}
رابطه مستقیم مشاهده شده است
\!\citep{hartungBetterItsReputation2019}
\!.
همچنین نمرات سه‌گانه تاریک پیشگوی رفتار نابهنجار در سطح
\textit{
    اجتماعی-روانشناختی
}
\LTRfootnote{
    Psychosocial
}
می‌باشد
\!\citep{murisMalevolentSideHuman2017}
\!.
رفتار ضد اجتماعی قوی‌ترین رابطه را با رفتار نابهنجار اجتماعی-روانشناختی دارد
($r = 0.29$)
\!. خودشیفتگی
($r = 0.24$)
و ماکیاولیسم
($r = 0.13$)
در رتبه‌های بعدی قرار می‌گیرند
\!.
در پژوهش حاضر میان رفتار به اشتراک‌گذاری اطلاعات شخصی دیگران و سه‌گانه تاریک رابطه مشاهده شد.

% ^ %%%%%%%%%%%%%%%%%%%%%%%%%%%%%%%%%%%%%%%%%%%%%%%%%%%%%%%%%%
% ^ %%%%%%%%%%%%%%%%%%%%%%%%%%%%%%%%%%%%%%%%%%%%%%%%%%%%%%%%%%
% ^ %%%%%%%%%%%%%%%%%%%%%%%%%%%%%%%%%%%%%%%%%%%%%%%%%%%%%%%%%%
\section{
  % \subsection{
  نتایج مربوط به
  \lr{SVO}
 }
% ^ %%%%%%%%%%%%%%%%%%%%%%%%%%%%%%%%%%%%%%%%%%%%%%%%%%%%%%%%%%
% ^ %%%%%%%%%%%%%%%%%%%%%%%%%%%%%%%%%%%%%%%%%%%%%%%%%%%%%%%%%%
% ^ %%%%%%%%%%%%%%%%%%%%%%%%%%%%%%%%%%%%%%%%%%%%%%%%%%%%%%%%%%
% \subsection{
% \section{
%   نتایج مربوط به کیفیت زندگی
%  }
% در پژوهش‌های قبلی میان عامل خود‌افشاگری و
% دامنه‌های کیفیت ارتباطات اجتماعی و کیفیت محیطی و
% نمره کل کیفت زندگی در پرسش‌نامه کیفیت زندگی روابطی مشاهده شده است
% \!\citep{chandraRelationshipPsychologicalMorbidity2003}
% \!.
% تاکنون شما در پایان‌نامه‌ای که مشغول نوشتن آن هستید، پاسخ چهار سؤال را داده‌اید:
% \begin{itemize}
% 	\item
% 	چرا تحقیق را انجام دادید؟ (مقدمه)
% 	\item
% 	دیگران در این زمینه‌ چه کارهایی کرده‌اند و تمایز کار شما با آنها؟ (مرور ادبیات)
% 	\item
% 	چگونه تحقیق را انجام دادید؟ (روش‌ها)
% 	\item
% 	چه از تحقیق به دست آوردید؟ (یافته‌ها)
% \end{itemize}
% حال زمان آن فرا رسیده که با توجه به تمامی مطالب ذکر شده، در نهایت به سؤال آخر پاسخ دهید:
% \begin{itemize}
% 	\item
% 	چه برداشتی از یافته‌های تحقیق کردید؟ (نتیجه‌گیری)
% \end{itemize}
% در واقع در این بخش، هدف، پاسخ به این سوال است که چه برداشتی از یافته‌ها کردید و این یافته‌ها چه فایده‌ای دارند؟
یافته‌های ما نشان می‌دهد که افراد با نمره سه‌گانه تاریک بالاتر تمایل کمتری برای به اشتراک گذاری دارند.
این نتیجه با یافته‌های پژوهش‌های محدودی که قبلا انجام  شده است مطابقت دارد.
% نتیجه‌گیری مختصری بنویسید. ارائهٔ داده‌ها، نتایج و یافته‌ها در فصل چهارم ارائه می‌شود. در این فصل تفاوت، تضاد یا تطابق بین نتایج تحقیق با نتایج دیگر محققان باید ذکر شود.
% \emph{تفسیر و تحلیل نتایج نباید بر اساس حدس و گمان باشد}،
% بلکه باید
% \textbf{برمبنای نتایج عملی استخراج‌شده}
% از تحقیق و یا
% \textbf{استناد به تحقیقات دیگران}
% باشد.
% با توجه به حجم و ماهیت تحقیق و با صلاحدید استاد راهنما، این فصل می‌تواند تحت عنوانی دیگر بیاید یا به دو فصل جداگانه با عناوین مناسب، تفکیک شود. اين فصل فقط باید به جمع‌بندی دست‌آوردهای فصل‌های سوم و چهارم محدود و از ذکر موارد جدید در آن خودداری شود. در عنوان این فصل، به جای کلمهٔ «تفسیر» می‌توان از واژگان «بحث» و «تحلیل» هم استفاده کرد. این فصل شاید مهم‌ترین فصل پایان‌نامه باشد.

% در این فصل خلاصه‌ای از یافته‌های تحقیق جاری ارائه می‌شود. این فصل می‌تواند حاوی یک مقدمه، شامل مروری اجمالی بر مراحل انجام تحقیق باشد (حدود یک صفحه). مطالب پاراگراف‌بندی شود و هر پاراگراف به یک موضوع مستقل اختصاص یابد. فقط به ارائهٔ یافته‌ها و دست‌آوردها بسنده شود و
% \emph{از تعمیم بی‌مورد نتایج خودداری شود.}
% تا حد امکان از ارائهٔ 
% \emph{جداول و نمودارها در این فصل اجتناب شود.}
% از ارائهٔ 
% \emph{عناوین کلی}
% در حوزهٔ تحقیق و قسمت پیشنهاد پیشنهاد تحقیقات آتی خودداری شود و کاملاً در چارچوب و زمينهٔ مربوط به تحقیق جاری باشد. این فصل حدود ۱۰-۱۵ صفحه است.

% \section{محتوا}
% به ترتیب شامل موارد زیر است:

% \subsection{جمع‌بندی}
% \section{جمع‌بندی}
% خلاصه‌ای از تمام یافته‌ها و دست‌آوردهای تحقیق جاری است.

% \subsection{نوآوری}
% \section{نوآوری}
% این قسمت، نوآوری تحقیق را بر اساس یافته‌های آن تشریح می‌کند. که دارای دو بخش اصلی است:
% \begin{enumerate}
% 	\item
% 	نوآوری تئوری، یعنی تمایز تئوریک کار با کارهای محققین قبلی.
% 	\item
% 	نوآوری عملی، یعنی توصیه‌های محقق به صنعت برای بهبود بخشیدن به کارها، بر اساس یافته‌های تحقیق.
% \end{enumerate}


% \subsection{محدودیت‌ها}
\section{محدودیت‌ها}
یک محدودیت اساسی  که سبب‌ساز ضرورت اجرا کردن آزمایش به صورت
پایلوت در دفعات متعدد شد، عدم دسترسی به آزمودنی‌ها بود. معمولا
در چنین پژوهش‌های میدانی که توسط آزمون‌های مبتنی بر وب
انجام می‌گیرد، از سرویس آمازون مکانیکال تورک و یا سرویس های مشابه مانند
\textit{
    \gls{Prolific}
}
استفاده می‌شود. این سرویس‌ها امکان به کار گرفتن آزمودنی‌ها با شرایط قابل قبول
از نظر سوابق قبلی آن‌ها، را امکان‌پذیر می‌سازند. در این پژوهش از
شبکه‌های اجتماعی  اینستاگرام و تلگرام برای جذب آزمودنی‌ها استفاده شد. در
نتیجه غربالگری افراد بر اساس شرایط قابل تایید
\textit{
    \gls{Exclusion criteria}
}
وجود نداشت.

با وجود این‌که از نظر فنی امکان علامت‌گذاری آزمودنی‌های بر ا ساس اینکه از لینک موجود در اینستاگرام وارد آزمون شده بودند و یا لینک
در کانال‌های تلگرام، در این آزمایش علامت‌گذاری انجام نشد و لذا امکان انجام تحلیل‌های آماری برای مقایسه رفتار آزمودنی‌های تلگرام و اینستاگرام
فراهم نشد.

بخشی از آزمودنی‌ها مخصوصا آنهایی که از لینک اینستاگرام برای ورود به آزمایش استفاده
کردند، از دوستان و آشنایان انجام دهنده آزمایش بوده‌اند. با توجه به اینکه اعتماد به انجام دهنده آزمایش نقش
به عنوان یک متغیر تعدیل کننده، می‌تواند بر روی متغیر‌های وابسته تاثیر بگذارد و رفتار آزمودنی را تغییر دهد،
در این آزمایش تاثیر آن بررسی و کنترل نشده‌است.

داده‌های مرتبط با نوع مرورگر، آدرس آی.پی و موقعیت بر اساس آی.پی، نوع کامپیوتر و یا گوشی هوشمند مورد استفاده
در آزمایش جمع‌آوری نشده اند، اما می‌توان در پژوهش‌های آینده با استفاده از لاگ سرویس دهنده وب و یا تعبیه کدهای جمع‌آوری کننده این داده‌ها
در خود آزمون می توان این داده‌ها را جمع‌آوری و تحلیل کرد.

با توجه به کدهای نوشته شده برای قسمت فرانت‌اند و بک‌اند آزمون مبتنی بر وب امکان بررسی زمان خروج آزمونی‌هایی که در میانه یک پرسش‌نامه،
ریزش داشته اند فراهم نبوده است و داده های موجود فقط امکان مشخص کردن زمان خروج آزمودنی ها در بین دو پرسش‌نامه جداگانه، را فراهم  می‌کنند.

% \subsection{محدودیت‌های ابزار سنجش نگرش به انواع اطلاعات خصوصی}
% \section*{محدودیت‌های ابزار سنجش نگرش به انواع اطلاعات خصوصی}
% در این پژوهش از برای سنجش
% {\textit{\gls{Attitude}}}
% نسب به ارزش انواع داده های خصوصی از دید کاربران
% (ضمیمه \ref{app:questionnaires})
% استفاده شده است.

% با وجود اینکه روش حراج تجربی برای سنجش باور افراد نسبت به ارزش اطلاعات دارای اعتبار بیشتری است، با توجه به محدودیت پرداخت
% به آزمودنی‌ها و انتخاب رویکرد اکتشافی برای انجام این پژوهش، از روش نظرسنجی به این منظور استفاده شد. در پژوهش‌های پیش رو می‌توان از روش حراج تجربی برای سنجش ارزش داده
% استفاده کرد. با توجه به اینکه در این روش آزمودنی‌ها می‌باید علم داده دانش کافی داشته باشند، پیشنهاد می‌شود از جامعه فعالان حوزه علم داده برای جمع‌آوری نمونه استفاده
% شود.
% \subsection{پیشنهادها}
\section{پیشنهادها}
با  توجه به اینکه  حریم خصوصی افراد یکی از متغیر‌های اصلی در زمان جمع آوری
داده بود، رفتار‌های متنوعی در آزمودنی‌ها حین انجام آزمایش به وجود
آمد. رها کردن انجام آزمایش یکی از این رفتار‌ها
بود. بررسی این رفتار به عنوان یک متغیر، می‌تواند مهم تلقی شود.

در این پژوهش از سه سطح 
\textit{
    \gls{Temptation}
}
به همراه یک 
\textit{
    \gls{Base line}
}
برای سنجش تغییر رفتار تحت تاثیر 
\textit{
    \gls{Temptation}
} 
استفاده شد. با توجه به اینکه پنج دسته تغییر رفتار تعریف شده است و برخی از این تغییر رفتار‌ها مانند رفتار
\textit{
    \gls{XNull}
}
به ندرت دیده می‌شوند، می‌توان تعداد دسته‌های تغییر رفتار را کاهش داد یا برای بدست آوردن اندازه اثر مناسب و تعداد کافی آزمودنی 
در هر یک از این دسته‌ها حجم نمونه را به حدود پانصد نفر افزایش یابد.

طراحی بخش سنجش تغییر رفتار بر اساس 
\textit{
    \gls{Pretest-Posttest Control Group Design}
}
انجام شد. برای سنجش 
\textit{
    \gls{Efficacy}
}
\textit{
    \gls{Temptation}
} 
بر روی تغییر 
% این بخش، عناوین و موضوعات پیشنهادی را برای تحقیقات آتی،
% \emph{بیشتر در زمينهٔ مورد بحث در آينده}
% ارائه می‌کند.
