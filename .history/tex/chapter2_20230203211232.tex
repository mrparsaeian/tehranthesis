% !TeX root=main.tex
\chapter{مروری بر مطالعات انجام شده}
%\thispagestyle{empty} 
% \section{مقدمه}
% هدف از اين فصل که با عنوان‌های  «مروری بر ادبیات موضوع%
% \LTRfootnote{Literature Review}»،
% «مروری بر منابع» و يا «مروری بر پیشینه تحقیق%
% \LTRfootnote{Background Research}»
% معرفی می‌شود، بررسی و طبقه‌بندی یافته‌های تحقیقات دیگر محققان در سطح دنیا و تعیین و شناسایی خلأهای تحقیقاتی است. آنچه را که تحقیق شما به دانش موجود اضافه می‌کند، مشخص کنید. طرح پیشینه تحقیق%
% \LTRfootnote{Background Information}
% یک مرور محققانه است و تا آنجا باید پیش برود که پیش‌زمینهٔ تاریخی مناسبی از تحقیق را بیان کند و جایگاه تحقیق فعلی را در میان آثار پیشین نشان دهد. برای این منظور منابع مرتبط با تحقیق را بررسی کنید، البته نه آنچنان گسترده که کل پیشینه تاریخی بحث را در برگیرد. برای نوشتن این بخش:
% \begin{itemize}
% 	\item
% 	دانستنی‌های موجود و پیش‌زمینهٔ تاریخی و وضعیت کنونی موضوع را چنان بیان کنید که خواننده بدون مراجعه به منابع پیشین، نتایج حاصل از مطالعات قبلی را درک و ارزیابی کند.
% 	\item
% 	نشان دهید که بر موضوع احاطه دارید. پرسش تحقیق را همراه بحث و جدل‌ها و مسائل مطرح شده بیان کنید و مهم‌ترین تحقیق‌های انجام شده در این زمینه را معرفی نمائید.
% 	\item
% 	ابتدا مطالب عمومی‌تر و سپس پژوهش‌های مشابه با کار خود را معرفی کرده و نشان دهید که تحقیق شما از چه جنبه‌ای با کار دیگران تشابه یا تفاوت دارد.
% 	\item
% 	اگر کارهای قبلی را خلاصه کرده‌اید، از پرداختن به جزئیات غیرضروری بپرهیزید. در عوض، بر یافته‌ها و مسائل روش‌شناختی مرتبط و نتایج اصلی تأکید کنید و اگر بررسی‌ها و منابع مروری عمومی دربارهٔ موضوع موجود است، خواننده را به آنها ارجاع دهید.
% \end{itemize}


% \section{مروری بر ادبیات موضوع}
% در این قسمت باید به کارهای مشابه دیگران در گذشته اشاره کرد و وزن بیشتر این قسمت بهتر است به مقالات ژورنالی سال‌های اخیر (۲ تا ۳ سال) تخصیص داده شود. به نتایج کارهای دیگران با ذکر دقیق مراجع باید اشاره شده و جایگاه و تفاوت تحقیق شما نیز با کارهای دیگران مشخص شود. استفاده از مقالات ژورنال‌های معتبر در دو یا سه سال اخیر، می‌تواند به اعتبار کار شما بیافزاید.

% \section{نتیجه‌گیری}
% ‌در نتیجه‌گیری آخر این فصل، با توجه به بررسی انجام شده بر روی مراجع تحقيق، بخش‌های قابل گسترش و تحقیق در آن حیطه و چشم‌اندازهای تحقیق مورد بررسی قرار می‌گیرند.	در برخی از تحقیقات، نتیجه نهایی فصل روش تحقیق، ارائهٔ یک چارچوب کار تحقیقی 
% \lr{(research framework)}
% است.

	\section{تعاریف، اصول و مبانی نظری}
	% این قسمت ارائهٔ خلاصه‌ای از دانش کلاسیک موضوع است. این بخش الزامی نیست و بستگی به نظر استاد راهنما دارد.


\ifTheEffectsOfDifferentPersonalData
	\textit{چاو، اوی و هربلاند}
	\LTRfootnote{Chua, H.N., J.S. Ooi, and A. Herbland}
	در سال ۲۰۲۱ اطلاعات شخصی را در ۶ دسته طبقه بندی کرده‌اند.
	\citep{chuaEffectsDifferentPersonal2021}
	جدول
	\eqref{tab:PICatAndChar}
	\begin{table}[ht]
		\caption{دسته‌ها و ویژگی‌های اطلاعت شخصی}
		\label{tab:PICatAndChar}
		\centering
		\onehalfspacing
		% \begin{tabular}{p{0.35\linewidth} | p{0.6\linewidth}}  
		\begin{tabularx}{\linewidth}{ r X }
			% \hline 
			دسته
			 &
			% درجه آزادی 
			% تبدیل مختصات &
			توضیح
			\\
			\hline
			سبک زندگی-رفتار(LB)
			 &
			% zotero://select/library/items/LCYS3GQ2
			% https://www.zotero.org/users/5038267/items/LCYS3GQ2
			%توضیح
			اطلاعت درباره ویژگی‌ها و سبک زندگی فرد که بر رابطه عاطفی یا
			اجتماعی، ترجیهات، عادات، باورها، یا دیدگاه‌های او اثر می‌گذارد.
			% مثال‌ها:
			% \lr{L1}:
			% باور
			% (
			% مانند باورهای مذهبی، باورهای فلسفی، افکار و غیره
			% )
			% % \hline 
			% \lr{L2}:
			% ترجیهات و علایق
			% (مانند نظرات، تمایلات، علایق، غذاهای مورد علاقه، رنگ‌ها، چیزهای
			% دوست‌داشتنی و دوست‌نداشتنی)
			% \lr{L3}:
			% رفتار
			% (مانند رفتار وبگردی، الگوی تماس‌ها، لینک‌های کلیک شده،
			% سبک زندگی رویکرد و غیره)
			% \lr{L4}:
			% خانواده/روابط عاطفی
			% (مانند ساختار خانواده، همشیرها، فرزندان، ازدواج‌ها، طلاق‌ها،
			% روابط عاطفی و غیره)
			\\
			اجتماعی-اقتصادی(SE)
			 &
			% ۳ & 
			% ویژگی دو &
			اطلاعاتی که سطح زندگی اقتصادی یا اجتماعی فرد را نشان می‌دهند یا
			می‌توان به وسیله این اطلاعات ویژگی‌های مزبور را استخراج کرد.
			% مثال‌ها;
			% \lr{S1}:
			% \lr{S1}:
			% \lr{S1}:
			\\
			ردیابی(T)
			 &
			اطلاعاتی که روش‌هایی را برای موقعیت‌یابی و تماس با فرد ایجاد می کند.
			% \hline 
			\\
			اقتصادی(F)
			 &
			اطلاعاتی که درآمد، حساب‌های مالی، اعتبار، توانایی خرید/خرج کردن، و
			دارایی‌های مورد تملک/اجاره شده/قرض گرفته شده را مشخص می کند.
			\\
			احراز هویت(A)
			 &
			اطلاعاتی که برای احراز هویت فرد به کار می‌روند.
			\\
			پزشکی-سلامت(MH)
			 &
			شرایط پزشکی  یا اطلاعات مرتبط با سلامت فرد.
			\\
		\end{tabularx}
	\end{table}
	% \begin{table}[ht]
	%     \caption{مدلهای تبدیل دیگر.}
	%     \label{tab:motionModelsCont}
	%     \centering
	%     \onehalfspacing
	%     \begin{tabularx}{\textwidth}{|r|c|l|X|}
	%         \hline نام مدل & درجه آزادی & تبدیل مختصات & توضیح \\ 
	%         \hline مشابهت & ۴ & $\begin{aligned} x'=sx\cos\theta - sy\sin\theta+t_x \\ y'=sx\sin\theta+sy\cos\theta+t_y  \end{aligned}$  & اقلیدسی+تغییرمقیاس \\ 		
	%         \hline آفین & ۶ & $\begin{aligned} x'=a_{11}x+a_{12}y+t_x \\ y'=a_{21}x+a_{22}y+t_y \end{aligned}$  & مشابهت+اریب‌شدگی \\
	%         \hline
	%     \end{tabularx}
	% \end{table}


	از سوی دیگر
\fi %TheEffectsOfDifferentPersonalData
\ifMultidimensionalNatureOfPrivacyRisksConceptualisationMeasurementAndImplicationsForDigitalServices
	در سال ۲۰۲۲
	\textit{
		\gls{Sabrina Karwatzki, Manuel Trenz and Daniel Veit}
	}
	 یک دسته‌بندی 
	\textit{
		\gls{Nomological}
	}
	و جامع از نگرانی‌های حریم خصوصی افراد بر اساس پژوهش های گذشته ارائه داده و اعتبار و روایی آن‌را مورد سنجش قرار‌داده‌اند
	\citep{karwatzkiMultidimensionalNaturePrivacy2022}
	\!.
	آنها ۷ دسته
	\textbf{
		خطر حریم خصوصی فیزیکی
		\lr{(PH)}
		\LTRfootnote{Physical privacy risk}
		،
		خطر حریم خصوصی اجتماعی
		\lr{(SO)}
		\LTRfootnote{Social privacy risk}
		،
		خطر حریم خصوصی مرتبط با منابع
		\lr{(RE)}
		\LTRfootnote{Resource-related privacy risk}
		،
		خطر حریم خصوصی روانشناختی
		\lr{(PS)}
		\LTRfootnote{Psychological privacy risk}
		،
		خطر حریم خصوصی مرتبط با تعقیب قانونی
		\lr{(PR)}
		\LTRfootnote{Prosecution-related privacy risk}
		،
		خطر حریم خصوصی مرتبط با شغل
		\lr{(CR)}
		\LTRfootnote{Career-related privacy risks}
		\textmd{و}
		خطر حریم خصوصی مرتبط با آزادی
		\lr{(FR)}
		\LTRfootnote{Freedom-related privacy risk}
	}
\fi %\ifMultidimensionalNatureOfPrivacyRisksConceptualisationMeasurementAndImplicationsForDigitalServices
را شناسایی کردند. ما
از این دسته‌بندی و توصیفاتی که در این پژوهش با دسته‌بندی‌های نامبرده مرتبط دانسته شده اند، استفاده کردیم.
همان‌طور که در بخش قبل گفته شد، چهارده 
مجموعه‌داده
بر اساس این دسته‌بندی توسط نویسنده پایان‌نامه، تعریف  شد. برای اینکه این مجموعه‌داده‌ها با
تعاریفی که کاربران در پژوهش 
\textit{
	\gls{Sabrina Karwatzki, Manuel Trenz and Daniel Veit}
}