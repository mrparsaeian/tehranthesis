% در این فایل، دستورها و تنظیمات مورد نیاز، آورده شده است.
%-------------------------------------------------------------------------------------------------------------------
% دستوراتی که پوشه پیش‌فرض زیرفایل‌های tex را مشخص می‌کند.
%\makeatletter
%\def\input@path{{./tex/}}
%\makeatother
% در ورژن جدید زی‌پرشین برای تایپ متن‌های ریاضی، این سه بسته، حتماً باید فراخوانی شود
\usepackage{amsthm,amssymb,amsmath}
% بسته‌ای برای تنطیم حاشیه‌های بالا، پایین، چپ و راست صفحه
\usepackage[top=40mm, bottom=40mm, left=25mm, right=35mm]{geometry}
% بسته‌‌ای برای ظاهر شدن شکل‌ها و تعیین آدرس تصاویر
\usepackage[final]{graphicx}
\graphicspath{{./img/}}
% بسته‌های مورد نیاز برای نوشتن کدها، رنگ‌آمیزی آنها و تعیین پوشهٔ کدها
\usepackage[final]{listings}
\usepackage[usenames,dvipsnames,svgnames,table]{xcolor}
\lstset{inputpath=./code/}
% بسته‌ای برای رسم کادر
\usepackage{framed}
% بسته‌‌ای برای چاپ شدن خودکار تعداد صفحات در صفحه «معرفی پایان‌نامه»
\usepackage{lastpage}
% بسته‌ٔ لازم برای: ۱. تغییر شماره‌گذاری صفحات پیوست. ۲. تصحیح باگ آدرس وب حاوی '%' در مراجع
\usepackage{etoolbox}

%%%%%%%%%%%%%%%%%%%%%%%%%%%%%%%%%%%%
%%% دستورات وابسته به استیل مراجع:
%% 
% اگر از استیل‌های
%  natbib (plainnat-fa، asa-fa، chicago-fa)
%   استفاده می‌کنید، خط زیر را فعال و بعدی‌اش را غیرفعال کنید.
\usepackage{natbib}
% \newcommand{\citelatin}[1]{\cite{#1}\LTRfootnote{\citeauthor*{#1}}}
\newcommand{\citeplatin}[1]{\citep{#1}\LTRfootnote{\citeauthor*{#1}}}
%% اگر از سایر استیل‌ها استفاده می‌کنید، خط بالا را غیرفعال و ۳ خط‌های زیر را فعال کنید.
% \let\citep\cite
% \let\citelatin\cite
% \let\citeplatin\cite
%%%%%%%%%%%%
% بررسی حالت پیش نویس
\usepackage{ifdraft}
\ifdraft
{%
	% بسته‌ٔ ایجاد لینک‌های رنگی با امکان جهش
	\usepackage[unicode=true,pagebackref=true,
		colorlinks,linkcolor=blue,citecolor=blue,final]{hyperref}
	%\usepackage{todonotes}
	\usepackage[firstpage]{draftwatermark}
	\SetWatermarkText{\ \ \ پیش‌نویس}
	\SetWatermarkScale{1.2}
}
{
	\usepackage[pagebackref=false,colorlinks,
		linkcolor=blue,citecolor=blue,urlcolor=blue]{hyperref}
	%\usepackage[disable]{todonotes} % final without TODOs
}

\usepackage[obeyDraft]{todonotes}
\setlength{\marginparwidth}{2cm}

% تعیین مشخصات فایل PDF
\hypersetup{
	pdftitle={My Thesis},
	pdfauthor={Sina Momken},
	pdfsubject={Master Thesis},
	pdfkeywords={keywords},
	pdfdirection={R2L}
}
%%%%%%%%%%%%
%%%
%  تصحیح باگ
%  :
%   اگر
%    در
%     مراجع،
% 	 آدرس
% 	  وب
% 	   حاوی
% 	    '%'
% 		 بوده و
% 		  pagebackref
% 		   فعال
% 		    باشد،
% 			 دستورات زیر باید بیایند:
%% برای استیل‌های
%  natbib
%   مثل 
%   plainnat-fa،
%    asa-fa،
%     chicago-fa

\makeatletter
\let\ORIG@BR@@lbibitem\BR@@lbibitem
\apptocmd\ORIG@BR@@lbibitem{\endgroup}{}{}
\def\BR@@lbibitem{\begingroup\catcode`\%=12 \ORIG@BR@@lbibitem}
\makeatother
%% برای سایر استیل‌ها
% \makeatletter
% \let\ORIG@BR@@bibitem\BR@@bibitem
% \apptocmd\ORIG@BR@@bibitem{\endgroup}{}{}
% \def\BR@@bibitem{\begingroup\catcode`\%=12 \ORIG@BR@@bibitem}
% \makeatother
%%%%%%%%%%%%%%%%%%%%%%%%%%%%%%%%%%%%

% بسته‌ لازم برای تنظیم سربرگ‌ها
\usepackage{fancyhdr}
%\usepackage{enumitem}
\usepackage{setspace}
% بسته‌های لازم برای نوشتن الگوریتم
\usepackage{algorithm}
\usepackage{algorithmic}
% بسته‌های لازم برای رسم بهتر جداول
\usepackage{tabulary}
\usepackage{tabularx}
% بسته‌های لازم برای رسم تنظیم بهتر شکل‌ها و زیرشکل‌ها
\usepackage[export]{adjustbox}
\usepackage{subfigure}
\usepackage[subfigure]{tocloft}
% بسته‌ای برای رسم نمودارها و نیز صفحه مالکیت اثر
\usepackage{tikz}
% بسته‌ای برای ظاهر شدن «مراجع» و «نمایه» در فهرست مطالب
\usepackage[nottoc]{tocbibind}
% دستورات مربوط به ایجاد نمایه
\usepackage{makeidx}
\makeindex
%%% بسته ایجاد واژه‌نامه با xindy
\usepackage[xindy,acronym,nonumberlist=true]{glossaries}
%  ^ برای وارد کردن پی دی اف پرسشنامه ها از سایت
% \usepackage{pdfpages}
% بسته زیر باگ ناشی از فراخوانی بسته‌های زیاد را برطرف می‌کند.
\usepackage{morewrites}
%  ^ اجرای آمار در لاتک
% \usepackage[gobble=auto]{pythontex}
% \usepackage{pgfplots}
\newcommand{\vari}[1]{#1}
\newcommand{\block}[1]{#1}
% ^ %%%%%%%%%%%%%%%%%%%%%%%%%%%
%%%%%%%%%%%%%%%%%%%%%%%%%%
% فراخوانی بسته زی‌پرشین (باید آخرین بسته باشد)، تعریف قلم فارسی و انگلیسی و مکان قلم‌ها
\usepackage[extrafootnotefeatures]{xepersian}



\settextfont[Path={./font/}, BoldFont={IRLotusICEE_Bold.ttf}, BoldItalicFont={IRLotusICEE_BoldIranic.ttf}, ItalicFont={IRLotusICEE_Iranic.ttf},Scale=1.2]{IRLotusICEE.ttf}

% LiberationSerif or FreeSerif as free equivalents of Times New Roman
\setlatintextfont[Path={./font/}, BoldFont={LiberationSerif-Bold.ttf}, BoldItalicFont={LiberationSerif-BoldItalic.ttf}, ItalicFont={LiberationSerif-Italic.ttf},Scale=1]{LiberationSerif-Regular.ttf}
%%%%%%%%%%%%%%%%%%%%%%%%%%
% چنانچه می‌خواهید اعداد در فرمول‌ها، انگلیسی باشد، خط زیر را غیرفعال کنید
\setdigitfont[Path={./font/}, Scale=1.2]{IRLotusICEE.ttf}
%%%%%%%%%%%%%%%%%%%%%%%%%%
% تعریف قلم‌های فارسی و انگلیسی اضافی برای استفاده در بعضی از قسمت‌های متن
\defpersianfont\titlefont[Path={./font/}, Scale=1]{IRTitr.ttf}
\setiranicfont[Path={./font/}, Scale=1.3]{IRLotusICEE_Iranic.ttf}				% ایرانیک، خوابیده به چپ
% \defpersianfont\nastaliq[Scale=1.2]{IranNastaliq}
\setmathsfdigitfont[Path={./font/}]{IRTitr.ttf}
%%%%%%%%%%%%%%%%%%%%%%%%%%
% راستچین شدن todonotes
\presetkeys{todonotes}{align=right,textdirection=righttoleft}{}
\makeatletter
\providecommand\@dotsep{5}
\def\listtodoname{فهرست کارهای باقیمانده}
\def\listoftodos{\noindent{\Large\vspace{10mm}\textbf{\listtodoname}}\@starttoc{tdo}}
\renewcommand{\@todonotes@MissingFigureText}{شکل}
\renewcommand{\@todonotes@MissingFigureUp}{شکل}
\renewcommand{\@todonotes@MissingFigureDown}{جاافتاده}
%  ^ for items in table by me https://tex.stackexchange.com/questions/150492/how-to-use-itemize-in-table-environment
\newcommand{\tabitem}{~~\llap{\textbullet}~~}
%  ^ for words in page
% \newcommand{\ATT}{\textit{\gls{Attitude}}}
% \newcommand{\BE}{\textit{\gls{Behavioral economics}}}
\newcommand{\TRA}{\textit{\gls{Theory of reasoned action}}}
\newcommand{\TPB}{\textit{\gls{Theory of planned behavior}}}
\newcommand{\SUBNORM}{\textit{\gls{Subjective norm}}}
\newcommand{\PBC}{\textit{\gls{Perceived Behavioral control}}}
\newcommand{\BI}{\textit{\glspl{Behavioral intention}}}
\newcommand{\CONFRAM}{\textit{\gls{Conceptual framework}}}
\newcommand{\RMQ}{\textit{\gls{The researcher made a questionnaire}}}
% \newcommand{\IP}{\textit{\gls{\gls{Information privacy}}}}
\newcommand{\pio}{\textit{\gls{\gls{Personal information of Others}}}}




\makeatother
% دستوری برای حذف کلمه «چکیده»
\renewcommand{\abstractname}{}
% دستوری برای حذف کلمه «abstract»
%\renewcommand{\latinabstract}{}
% دستوری برای تغییر نام کلمه «اثبات» به «برهان»
\renewcommand\proofname{\textbf{برهان}}
% دستوری برای تغییر نام کلمه «کتاب‌نامه» به «مراجع»
\renewcommand{\bibname}{مراجع}
% دستوری برای تعریف واژه‌نامه انگلیسی به فارسی
\newcommand\persiangloss[2]{#1\dotfill\lr{#2}\\}
% دستوری برای تعریف واژه‌نامه فارسی به انگلیسی 
\newcommand\englishgloss[2]{#2\dotfill\lr{#1}\\}
% تعریف دستور جدید «\پ» برای خلاصه‌نویسی جهت نوشتن عبارت «پروژه/پایان‌نامه/رساله»
\newcommand{\پ}{پروژه/پایان‌نامه/رساله }

%\newcommand\BackSlash{\char`\\}

%%%%%%%%%%%%%%%%%%%%%%%%%%
\SepMark{-}

% تعریف و نحوه ظاهر شدن عنوان قضیه‌ها، تعریف‌ها، مثال‌ها و ...
\theoremstyle{definition}
\newtheorem{definition}{تعریف}[section]
% \theoremstyle{theorem}
\newtheorem{theorem}[definition]{قضیه}
\newtheorem{lemma}[definition]{لم}
\newtheorem{proposition}[definition]{گزاره}
\newtheorem{corollary}[definition]{نتیجه}
\newtheorem{remark}[definition]{ملاحظه}
\theoremstyle{definition}
\newtheorem{example}[definition]{مثال}

%\renewcommand{\theequation}{\thechapter-\arabic{equation}}
%\def\bibname{مراجع}
\numberwithin{algorithm}{chapter}
\def\listalgorithmname{فهرست الگوریتم‌ها}
\def\listfigurename{فهرست تصاویر}
\def\listtablename{فهرست جداول}

%%%%%%%%%%%%%%%%%%%%%%%%%%%%
%%% دستورهایی برای سفارشی کردن سربرگ صفحات:
%\newcommand{\SetHeader}[1]{
% دستور زیر معادل با گزینه twoside است.
%\csname@twosidetrue\endcsname
\pagestyle{fancy}
%% دستورات زیر سبک صفحات fancy را تغییر می‌دهد:
% O=Odd, E=Even, L=Left, R=Right
% در صورت oneside بودن، عنوان فصل، سمت چپ ظاهر می‌شود.
\fancyhead{}
\fancyhead[OL]{\small\leftmark}
\fancyhead[ER]{\small\leftmark}
% شکل‌دهی شماره و عنوان فصل در سربرگ
%\renewcommand{\chaptermark}[1]{\markboth{\thechapter-\ #1}{}}
%}
%%%%%%%%%%%%%%%%%%%%%%%%%%%%
%\def\MATtextbaseline{1.5}
%\renewcommand{\baselinestretch}{\MATtextbaseline}
\doublespacing
%%%%%%%%%%%%%%%%%%%%%%%%%%%%%
% دستوراتی برای اضافه کردن کلمه «فصل» در فهرست مطالب

\newlength\mylenprt
\newlength\mylenchp
\newlength\mylenapp

\renewcommand\cftpartpresnum{\partname~}
\renewcommand\cftchappresnum{\chaptername~}
\renewcommand\cftchapaftersnum{:}

\settowidth\mylenprt{\cftpartfont\cftpartpresnum\cftpartaftersnum}
\settowidth\mylenchp{\cftchapfont\cftchappresnum\cftchapaftersnum}
\settowidth\mylenapp{\cftchapfont\appendixname~\cftchapaftersnum}
\addtolength\mylenprt{\cftpartnumwidth}
\addtolength\mylenchp{\cftchapnumwidth}
\addtolength\mylenapp{\cftchapnumwidth}

\setlength\cftpartnumwidth{\mylenprt}
\setlength\cftchapnumwidth{\mylenchp}

\makeatletter
{\def\thebibliography#1{\chapter*{\refname\@mkboth
		  {\uppercase{\refname}}{\uppercase{\refname}}}\list
		{[\arabic{enumi}]}{\settowidth\labelwidth{[#1]}
			\rightmargin\labelwidth
			\advance\rightmargin\labelsep
			\advance\rightmargin\bibindent
			\itemindent -\bibindent

			\listparindent \itemindent
			\parsep \z@
			\usecounter{enumi}}
		\def\newblock{}
		\sloppy
		\sfcode`\.=1000\relax}}

%اگر مایلید در شماره گذاری حرفی و ابجد به جای آ از الف استفاده شود دستورات زیر را فعال کنید.   
%\def\@Abjad#1{%
%  \ifcase#1\or الف\or ب\or ج\or د%
%           \or هـ\or و\or ز\or ح\or ط%
%           \or ی\or ک\or ل\or م\or ن%
%           \or س\or ع\or ف\or ص%
%           \or ق\or ر\or ش\or ت\or ث%
%            \or خ\or ذ\or ض\or ظ\or غ%
%            \else\@ctrerr\fi}
%
% \def\abj@num@i#1{%
%   \ifcase#1\or الف\or ب\or ج\or د%
%            \or هـ‍\or و\or ز\or ح\or ط\fi

%   \ifnum#1=\z@\abjad@zero\fi}   
%  
%   \def\@harfi#1{\ifcase#1\or الف\or ب\or پ\or ت\or ث\or

% ج\or چ\or ح\or خ\or د\or ذ\or ر\or ز\or ژ\or س\or ش\or ص\or ض\or ط\or ظ\or ع\or غ\or

% ف\or ق\or ک\or گ\or ل\or م\or ن\or و\or ه\or ی\else\@ctrerr\fi}

%
\makeatother

%%% امکان درج کد در سند
% در این قسمت رنگ، قلم و قالب‌بندی قسمت‌های مختلف یک کد تعیین می‌شود. 
\lstdefinestyle{myStyle}{
	basicstyle=\ttfamily, % whole listing /w verbatim font
	keywordstyle=\color{blue}\bfseries, % bold black keywords
	identifierstyle=, % nothing happens
	commentstyle=\color{LimeGreen}, % green comments
	stringstyle=\ttfamily\color{red}, % red typewriter font for strings
	showstringspaces=false % no special string spaces
	breaklines=true,
	breakatwhitespace=false,
	numbers=right, % line number formats
	numberstyle=\footnotesize\lr,
	numbersep=-10pt,
	frame=single,
	captionpos=b,
	captiondirection=RTL
}
\lstset{style=myStyle} % command to set default style
\def\lstlistingname{\rl{برنامه}}


% for numbering subsubsections
\setcounter{secnumdepth}{3}
%to include subsubsections in the table of contents
\setcounter{tocdepth}{3}

