% !TeX root=main.tex
% در این فایل، عنوان پایان‌نامه، مشخصات خود، متن تقدیمی‌، ستایش، سپاس‌گزاری و چکیده پایان‌نامه را به فارسی، وارد کنید.
% توجه داشته باشید که جدول حاوی مشخصات پروژه/پایان‌نامه/رساله و همچنین، مشخصات داخل آن، به طور خودکار، درج می‌شود.
%%%%%%%%%%%%%%%%%%%%%%%%%%%%%%%%%%%%
% دانشگاه خود را وارد کنید
\university{دانشگاه تهران}
% پردیس دانشگاهی خود را اگر نیاز است وارد کنید (مثال: فنی، علوم پایه، علوم انسانی و ...)
\college{علوم پایه}
% دانشکده، آموزشکده و یا پژوهشکده  خود را وارد کنید
\faculty{دانشکده روانشناسی و علوم تربیتی}
% گروه آموزشی خود را وارد کنید (در صورت نیاز)
\department{گروه علوم شناختی}
% رشته تحصیلی خود را وارد کنید
\subject{علوم شناختی}
% گرایش خود را وارد کنید
\field{روانشناسی شناختی}
% عنوان پایان‌نامه را وارد کنید
\title{بررسی تصمیم‌گیری در به اشتراک گذاری اطلاعات خصوصی دیگران}
% نام استاد(ان) راهنما را وارد کنید
\firstsupervisor{دکتر مجیدی نیلی احمد‌آبادی}
\firstsupervisorrank{استاد}
\secondsupervisor{دکتر حسین وهابی}
\secondsupervisorrank{استادیار}
% نام استاد(دان) مشاور را وارد کنید. چنانچه استاد مشاور ندارید، دستورات پایین را غیرفعال کنید.
% \firstadvisor{دکتر مشاور اول}
% \firstadvisorrank{استادیار}
%\secondadvisor{دکتر مشاور دوم}
% نام داوران داخلی و خارجی خود را وارد نمایید.
\internaljudge{دکتر  رضا رستمی}
\internaljudgerank{استاد}
\externaljudge{دکتر محمدرضا ابوالقاسمی دهاقانی}
\externaljudgerank{استادیار}
\externaljudgeuniversity{دانشگاه تهران}
% نام نماینده کمیته تحصیلات تکمیلی در دانشکده \ گروه
\graduatedeputy{بابک نجاراعرابی}
\graduatedeputyrank{استاد}
% نام دانشجو را وارد کنید
\name{محمد رسول}
% نام خانوادگی دانشجو را وارد کنید
\surname{پارسائیان}
% شماره دانشجویی دانشجو را وارد کنید
\studentID{510697022}
% تاریخ پایان‌نامه را وارد کنید
\thesisdate{بهمن ۱۴۰۱}
% به صورت پیش‌فرض برای پایان‌نامه‌های کارشناسی تا دکترا به ترتیب از عبارات «پروژه»، «پایان‌نامه» و «رساله» استفاده می‌شود؛ اگر  نمی‌پسندید هر عنوانی را که مایلید در دستور زیر قرار داده و آنرا از حالت توضیح خارج کنید.
%\projectLabel{پایان‌نامه}

% به صورت پیش‌فرض برای عناوین مقاطع تحصیلی کارشناسی تا دکترا به ترتیب از عبارت «کارشناسی»، «کارشناسی ارشد» و «دکتری» استفاده می‌شود؛ اگر نمی‌پسندید هر عنوانی را که مایلید در دستور زیر قرار داده و آنرا از حالت توضیح خارج کنید.
%\degree{}
%%%%%%%%%%%%%%%%%%%%%%%%%%%%%%%%%%%%%%%%%%%%%%%%%%%%
%% پایان‌نامه خود را تقدیم کنید! %%
\dedication
{
{\Large تقدیم به:}\\
\begin{flushleft}{
	\huge
	% همسر \\
	% \vspace{7mm}
	% و\\
	% \vspace{7mm}
	مادر و پدرم
}
\end{flushleft}
}
%% متن قدردانی %%
%% ترجیحا با توجه به ذوق و سلیقه خود متن قدردانی را تغییر دهید.
\acknowledgement{
قدردان عالم هستی‌ام که لحظه‌ای و ذره‌ای در بیکران زمان و مکان به من هوشیاری درک خویش را اعطا نمود.
در آغاز وظیفه‌  خود  می‌دانم از زحمات بی‌دریغ اساتید  راهنمای خود،  جناب آقای دکتر مجید نیلی و دکتر حسین وهابی، صمیمانه تشکر و  قدردانی کنم که در طول انجام این پایان‌نامه با نهایت صبوری همواره راهنما و مشوق من بودند و قطعاً بدون راهنمایی‌های ارزنده‌ ایشان، این مجموعه به انجام نمی‌رسید.

%از همکاری و مساعدت‌های دکتر ... مسئول تحصیلات تکمیلی و سایر کارکنان دانشکده بویژه سرکار خانم ... کمال تشکر را دارم.

با سپاس بی‌دریغ خدمت دوستان گران‌مایه‌ام، آقایان آرین یزدان‌پناه و دکتر محمد رضا کاظمی در آزمایشگاه سیستم‌های شناختی و پژوهشکده فن‌آوری‌های همگرا، که با همفکری مرا صمیمانه و مشفقانه یاری داده‌اند.

و در پایان، بوسه می‌زنم بر دستان خداوندگاران مهر و مهربانی، پدر و مادر عزیزم و ستایش می‌کنم وجود ارزش‌مندشان را و تشکر می‌کنم از خانواده عزیزم به پاس عاطفه سرشار و گرمای امیدبخش وجودشان، که بهترین پشتیبان من بودند.
}
%%%%%%%%%%%%%%%%%%%%%%%%%%%%%%%%%%%%
%چکیده پایان‌نامه را وارد کنید
\fa-abstract{
% \lr{tehran-thesis}
اشتراک اطلاعات خصوصی دیگران توسط نهادهای جمع‌آوری‌کننده این اطلاعات، تاثیر زیادی در فرایند‌های اجتماعی به خصوص در سال‌های اخیر داشته‌است. در این پایان‌نامه تلاش کردیم که ارتباط میان 
ویژگی‌های شخصیتی تاریک شامل، ماکیاولیسم، روا‌ن‌پریشی و خودشیفتگی را در کنار رویکرد ارزش اجتماعی، با رفتار به اشتراک گذاری اطلاعات خصوصی دیگران بررسی کنیم.
برای سنجش تاثیر تطمیع بر روی تغییر رفتار، از طرح پژوهشی پیش‌آزمون-پس‌آزمون استفاده شد. در بخش اول
\!(معرفی اولیه)
از آزمودنی‌ها درخواست شد تا اطلاعات خصوصی شامل نام و نام خانوادگی و شماره تلفن یک فرد دیگر را برای شرکت در آزمایش
معرفی کنند. در بخش دوم یک پرسشنامه جهت دریافت اطلاعات دموگرافیک و در بخش سوم پرسشنامه سه‌گانه تاریک به آزمودنی ارائه شد و سپس در بخش چهارم
\!(معرفی ثانویه)،
یک بار دیگر 
درخواست معرفی افراد جدید برای شرکت در آزمایش مطرح شد. در این بخش آزمودنی‌ها به صورت تصادفی در ۴ گروه وارد شدند و به هر گروه درخواست تطمیع کننده با سطح متفاوتی ارايه شد.
در پایان رویکرد ارزش اجتماعی شرکت کنندگان اندازه‌گیری شد. نتایج نشان می‌دهند که کسب نمره بالاتر
سه‌گانه تاریک با تصمیم برای عدم به اشتراک‌گذاری در مرحله معرفی اولیه و تصمیم به معرفی در شرایط وجود تطمیع در مرحله معرفی ثانویه ارتباط دارد.
از میان کسانی که آزمون رویکرد ارزش اجتماعی را علاوه بر ۴ مرحله اول آزمایش انجام داده بودند افراد با نمره سه‌گانه تاریک بالاتر بیشتر در گروه 
فردگرا و با نمره پایینتر در گروه جامعه‌پسند قرار می‌گیرند.
}
% کلمات کلیدی پایان‌نامه را وارد کنید
\keywords{
	سه‌گانه تاریک،رویکرد ارزش اجتماعی، اشتراک ثانویه اطلاعات، اطلاعات خصوصی،تصمیم‌گیری
	}
% انتهای وارد کردن فیلد‌ها
%%%%%%%%%%%%%%%%%%%%%%%%%%%%%%%%%%%%%%%%%%%%%%%%%%%%%%
% ابتدای درج صفحات مختلف
\titlePage
% بررسی حالت پیش‌نویس
\ifoptiondraft{}{% 
	\besmPage
	\titlePage
	\davaranPage
%%%%%%%%%%%%%%%%%%%%%%%%%%%%
% در این قسمت اسامی اساتید راهنما، مشاور و داور باید به صورت دستی وارد شوند
%\renewcommand{\arraystretch}{1.2}


%%%%%%%%%%%%%%%%%%%%%%%%%%%
	\esalatPage
	\mojavezPage
% چنانچه مایل به چاپ صفحات «تقدیم»، «نیایش» و «سپاس‌گزاری» در خروجی نیستید، خط‌های زیر را با گذاشتن ٪  در ابتدای آنها غیرفعال کنید.
	\taghdimPage
	\ghadrdaniPage
} % end ifoptiondraft
\abstractPage
% شروع درج فهرست‌ها
\newpage\clearpage
\pagenumbering{harfi} % آ، ب، ...
\tableofcontents \newpage
% بررسی حالت پیش‌نویس برای بقیه فهرست‌ها
\ifoptiondraft{
	\listoftodos
}{%
	\listoffigures \newpage
	\listoftables  \newpage
	% \addcontentsline{toc}{chapter}{\listalgorithmname}
	\listofalgorithms %\newpage
	\addcontentsline{toc}{chapter}{\lstlistlistingname}
	\lstlistoflistings \newpage
	\printacronyms
} % end ifoptiondraft

\pagestyle{fancy}