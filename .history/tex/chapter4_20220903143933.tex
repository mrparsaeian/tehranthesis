% !TeX root=main.tex
\chapter{نتایج}
%\thispagestyle{empty} 
\label{chap:results}

% ارائهٔ داده‌ها، نتایج، تحلیل و تفسیر اولیهٔ آنها در این فصل ارائه می‌شود. در ارائهٔ نتایج با توجه به راهنمای كلی نگارش فصل‌ها، تا حد امکان، ترکیبی از نمودار و جدول استفاده شود. با توجه به حجم و ماهیت تحقیق و با صلاحدید استاد راهنما، اين فصل می‌تواند تحت عنوانی دیگر بیاید. در صورتی که حجم داده‌ها زیاد باشد، بهتر است به صورت نمودار یا در قالب ضمیمه ارائه نشده و فقط نمونه‌ها در متن آورده شود. در این فصل باید به سوالات تحقیق، عطف به یافته‌های محقق، پاسخ داده شود. اگر تحقیق دارای آزمون فرض باشد، پذیرش یا عدم پذیرش فرضیه‌ها در این فصل گزارش می‌شود. این فصل حدود ۴۰ صفحه است.

% در این بخش به سوالات تحقیق، بر اساس داده‌ها و یافته‌های محقق، پاسخ داده می‌شود. داده‌ها با فرمت مناسبی ارائه می‌شوند؛ مدل (ها) اجرا شده و نتیجه آن مشخص می‌شود.
\subsection{سیاهه ارزش‌گذاری مجموعه‌داده}

در این پژوهش از
% سیاهه ارزش‌گذاری مجموعه‌داده
{\textit{\gls{Dataset valuation invetory}}}
که یک
{\textit{\gls{The researcher made a questionnaire}}}
است برای اندازه‌گیری
\gls{Attitude}
نسبت به ارزش دسته‌های مختلف اطلاعات شخصی، استفاده شده است.
\textit{\gls{Trial}}
هر یک از آزمودنی‌ها در هر
\textit{\gls{Trial}}
به
\textit{\gls{Dataset}}
ارائه شده عددی از بازه صفر تا ۱۰۰ نسبت داده است. این عدد به عنوان مقیاسی از ارزش ذهنی
\textit{\gls{Subjective value}}
دسته‌ای از اطلاعات که
\textit{\gls{Trial}}
به آن تعلق دارد، تلقی می‌شود

زمان شروع
\textit{\gls{Online survey}}
\textit{\gls{Consent}}
آزمودنی‌ها به طور تصادفی و از طریق کد
\textit{\gls{JavaScript}}
اجرا شده در
\textit{\gls{Browser}}
به دو دسته تقسیم شدند.
\subsection{ویژگی‌های نمونه}
تعداد کل آزمودنی ها
$\InitialSampleSize$
نفر،
با بازه سنی
\ageMin
تا
\ageMax
و
میانگین
\sampleAgeMean
و انحراف استاندارد
\sampleAgeSD
بود
\!.
از این میان
\SampleSizeMale
نفر مذکر و
\SampleSizeFemale
مونث بودند.
بعد از حذف داده‌های مربوط به آزمودنی‌هایی که اطلاعات مخدوش یا غیر قابل استفاده داشتند، تعداد
\CleanedSampleSize
باقی ماندند.

\subsection{نتایج آزمون جهت‌گیری ارزش اجتماعی}
از میان همه شرکت کنندگان
\noOfIndividualisticParticipants
نفر در دسته
\textit{
    \gls{Individualistic}
}
،
\noOfCompetitiveParticipants
نفر در دسته
\textit{
    \gls{Competitive}
}
،
\noOfCooperativeParticipants
نفر در دسته
همکاری‌کننده
و
\noOfAltruisticParticipants
نفر در دسته
دیگر‌خواه
قرار داشتند.

میانگین کلی نمره‌ای که همه آزمودنی‌ها در هر دو گروه به ۱۴ سوال هر دو دسته
نمرات از دید خود
\!(باور به ارزش اطلاعات)
\meanOfSelfWTPAllTwoParticipantGroupsAllTwoQuestionSection
با انحراف استاندارد
\SDOfSelfWTPAllTwoParticipantGroupsAllTwoQuestionSection
و از دید دیگران
\meanOfOtherWTPAllTwoParticipantGroupsAllTwoQuestionSection
\!(باور هنجاری به ارزش اطلاعات)
با انحراف استاندارد
\SDOfOtherWTPAllTwoParticipantGroupsAllTwoQuestionSection
بود.

\subsection{پایایی پرسشنامه سیاهه ارزش‌گذاری مجموعه‌داده}
میان نمرات به نیمه اول این پرسشنامه برای اندازه گیری
باور به ارزش اطلاعات
\!(ارزش‌گذاری از دید خود)
\meanOfSelfWTPAllTwoParticipantGroupFirstQuestionSection
با انحراف معیار
\SDOfSelfWTPAllTwoParticipantGroupsFirstQuestionSection
و نیمه دوم
\meanOfOtherWTPAllTwoParticipantGroupsSecondQuestionSection
\SDOfOtherWTPAllTwoParticipantGroupsSecondQuestionSection
بود. با توجه به وجود
%  عدم وجود
همبستگی میان این دو نیمه
(
$
    r=
    \PiersonrValueForCorrelationBetweenFirstAndSecondPartOfQuestionsForSelfValuation
    ,
    P=
    \PvalueForCorrelationBetweenFirstAndSecondPartOfQuestionsForSelfValuation
$
)
این آزمون برای اندازه گیری باور به ارزش اطلاعات از دید خود در گروه‌های هفت‌گانه دارای پایایی درونی با
روش نیمه‌سازی پرسشنامه است.
همچنین با توجه به وجود
%  عدم وجود
همبستگی میان این دو نیمه
(
$
    \!r=
    \!\PiersonRValueForCorrelationBetweenFirstAndSecondPartOfQuestionsforOtherValuation
    \!,
    P=
    \!\PvalueForCorrelationBetweenFirstAndSecondPartOfQuestionsForOtherValuation
$
)
این آزمون برای اندازه گیری باور به ارزش اطلاعات از دید دیگری در گروه‌های هفت‌گانه دارای پایایی درونی با
روش
\textit{
    \gls{Split Half}
}
پرسشنامه است.


میانگین نمرات آزمودنی ها به دسته اول سوالات
\section{اعتبارسنجی}
% از طریق مقایسهٔ نتایج با نتایج کارهای دیگران، استفاده از روش‌های تحلیل پایائی
% \lr{(reliability)}
% و اعتبار
% \lr{(validity)}،
% نظرگیری از خبرگان
% \lr{(expert judgment or feedback)}
% و یا
% \lr{triangulation}
% انجام می‌شود.

