% !TeX root=main.tex
% در این فایل، عنوان پایان‌نامه، مشخصات خود و چکیده پایان‌نامه را به انگلیسی، وارد کنید.

%%%%%%%%%%%%%%%%%%%%%%%%%%%%%%%%%%%%
\baselineskip=.6cm
\begin{latin}
\latinuniversity{University of Tehran}
\latincollege{Faculty of Psychology and Education}
\latinfaculty{Faculty of Psychology and Education}
\latindepartment{Cognitive Science}
\latinsubject{Cognitive Science}
\latinfield{Cognitive Psychology}
\latintitle{Investigation of decision-making in sharing personal information of others}
\firstlatinsupervisor{Dr. Majid Nili Ahmadabadi}
\secondlatinsupervisor{Dr. Abdol-Hossein Vahabie}
% \firstlatinadvisor{Mohamad Rasoul Parsaeian}
%\secondlatinadvisor{Second Advisor}
\latinname{Mohamad Rasoul}
\latinsurname{Parsaeian}
\latinthesisdate{February 2023}
\latinkeywords{secondary use of information, personal information, decision-making, dark triad, social value orientation}
\en-abstract{
    The sharing of personal information of others by the institutions that collect this information has a great impact on the social processes
especially in recent years. In this thesis, we tried to establish the relationship between the features
of the dark personality includes Machiavellian, psychopathy and narcissism, along with the  social value orientation, with behavior
of sharing of personal information of others. To measure the effect of bribery on behavior change, from
A pre-test-post-test research design was used. In the first part (initial introduction), participants were asked to
decide to enter personal information including first and last name and phone number of one to five other people
(pre-test). In the second part, a questionnaire to get demographic information and in the third part, dark triad inventory was presented to the participant.
Then in the fourth section (secondary introduction), the same forms as the pre-test section were presented
and we asked the subject to complete the list of new participants (post-test). In the pre-test phase, all
the subjects were asked to introduce new people as new participants in the experiment and to help with the 
research, but in the post-test section, the subjects were randomly entered into 4 groups and to each group
people were presented with different level of temptation. At the end, the participant's social value orientation
was measured. The results show a correlation between a higher score of the dark triad and the decision not to share information.
in the initial introduction stage. On the other hand  decision to introduce new participants is related to the presence of temptation in the secondary introduction stage.
 Among those who had done the social value approach test in addition to the first four stages of the test, people with a higher dark triad score
were mostly individualists, and those with a lower score were pre-social.
}
\latinTitlePage
\end{latin}
